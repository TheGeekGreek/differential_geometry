%%%%%%%%%%%%%%%%%%%%%%%%%%%%%%%%%%%%%%%%%%%%%%%%%%%%%%%%%%%%%%%%%%%%%%%%%%
%Author:																 %
%-------																 %
%Yannis Baehni at University of Zurich									 %
%baehni.yannis@uzh.ch													 %
%																		 %
%Version log:															 %
%------------															 %
%06/02/16 . Basic structure												 %
%04/08/16 . Layout changes including section, contents, abstract.		 %
%%%%%%%%%%%%%%%%%%%%%%%%%%%%%%%%%%%%%%%%%%%%%%%%%%%%%%%%%%%%%%%%%%%%%%%%%%

%Page Setup
\documentclass[
	12pt, 
	oneside, 
	a4paper,
	reqno,
	final
]{amsbook}

\usepackage[
	left = 3cm, 
	right = 3cm, 
	top = 3cm, 
	bottom = 3cm
]{geometry}

%Headers and footers


%Title
\usepackage[foot]{amsaddr}
\usepackage{newtxtext}
\usepackage[subscriptcorrection,nofontinfo,mtpcal,mtphrb]{mtpro2}
\usepackage{mathtools}
\usepackage{bm}
\usepackage{xspace}
\usepackage[all]{xy}
\usepackage{tikz-cd}
\makeatletter
\def\@textbottom{\vskip \z@ \@plus 1pt}
\let\@texttop\relax
\usepackage{etoolbox}
\patchcmd{\abstract}{\scshape\abstractname}{\textbf{\abstractname}}{}{}
\usepackage{chngcntr}
\counterwithout{figure}{chapter}
%Section, subsection and subsubsection font
%------------------------------------------
	\renewcommand{\@secnumfont}{\bfseries}
	\renewcommand\section{\@startsection{section}{1}%
  	\z@{.7\linespacing\@plus\linespacing}{.5\linespacing}%
  	{\normalfont\bfseries\boldmath\centering}}
	\renewcommand\subsection{\@startsection{subsection}{2}%
    	\z@{.5\linespacing\@plus.7\linespacing}{-.5em}%
    	{\normalfont\bfseries\boldmath}}%
	\renewcommand\subsubsection{\@startsection{subsubsection}{3}%
    	\z@{.5\linespacing\@plus.7\linespacing}{-.5em}%
    	{\normalfont\bfseries\boldmath}}%

		\renewenvironment{proof}{\textit{Proof}.}{\hfill\qedsymbol}

%ToC
%---
\makeatletter
\setcounter{tocdepth}{3}
% Add bold to \chapter titles in ToC and remove . after numbers
\renewcommand{\tocchapter}[3]{%
  	\indentlabel{\@ifnotempty{#2}{\bfseries\ignorespaces#1 #2: }}\bfseries#3}
\renewcommand{\tocappendix}[3]{%
  	\indentlabel{\@ifnotempty{#2}{\bfseries\ignorespaces#1 #2: }}\bfseries#3}
% Remove . after numbers in \section and \subsection
\renewcommand{\tocsection}[3]{%
  	\indentlabel{\@ifnotempty{#2}{\ignorespaces#1 #2\quad}}#3}
\renewcommand{\tocsubsection}[3]{%
  	\indentlabel{\@ifnotempty{#2}{\ignorespaces#1 #2\quad}}#3}
\let\tocsubsubsection\tocsubsection% Update for \subsubsection
%...
\newcommand\@dotsep{4.5}
\def\@tocline#1#2#3#4#5#6#7{\relax
  \ifnum #1>\c@tocdepth % then omit
  \else
    \par \addpenalty\@secpenalty\addvspace{#2}%
    \begingroup \hyphenpenalty\@M
    \@ifempty{#4}{%
      \@tempdima\csname r@tocindent\number#1\endcsname\relax
    }{%
      \@tempdima#4\relax
    }%
    \parindent\z@ \leftskip#3\relax \advance\leftskip\@tempdima\relax
    \rightskip\@pnumwidth plus1em \parfillskip-\@pnumwidth
    #5\leavevmode\hskip-\@tempdima{#6}\nobreak
    \leaders\hbox{$\m@th\mkern \@dotsep mu\hbox{.}\mkern \@dotsep mu$}\hfill
    \nobreak
    \hbox to\@pnumwidth{\@tocpagenum{\ifnum#1=0\bfseries\fi#7}}\par% <-- \bfseries for \chapter page
    \nobreak
    \endgroup
  \fi}
\AtBeginDocument{%
\expandafter\renewcommand\csname r@tocindent0\endcsname{0pt}
}
\def\l@subsection{\@tocline{2}{0pt}{2.5pc}{5pc}{}}
\def\l@subsubsection{\@tocline{2}{0pt}{4.5pc}{5pc}{}}
\makeatother

\advance\footskip0.4cm
\textheight=54pc    %a4paper
\textheight=50.5pc %letterpaper
\advance\textheight-0.4cm
\calclayout

%Font settings
%\usepackage{anyfontsize}
%Footnote settings
\usepackage{footmisc}
%	\renewcommand*{\thefootnote}{\fnsymbol{footnote}}
\usepackage{commath}
%Further math environments
%Further math fonts (loads amsfonts implicitely)
%Redefinition of \text
%\usepackage{amstext}
\usepackage{upref}
%Graphics
%\usepackage{graphicx}
%\usepackage{caption}
%\usepackage{subcaption}
%Frames
\usepackage{mdframed}
\allowdisplaybreaks
%\usepackage{interval}
\newcommand{\toup}{%
  \mathrel{\nonscript\mkern-1.2mu\mkern1.2mu{\uparrow}}%
}
\newcommand{\todown}{%
  \mathrel{\nonscript\mkern-1.2mu\mkern1.2mu{\downarrow}}%
}
\AtBeginDocument{\renewcommand*\d{\mathop{}\!\mathrm{d}}}
\renewcommand{\Re}{\operatorname{Re}}
\renewcommand{\Im}{\operatorname{Im}}
\DeclareMathOperator\Log{Log}
\DeclareMathOperator\Arg{Arg}
\DeclareMathOperator\id{id}
\DeclareMathOperator\sech{sech}
\DeclareMathOperator\Aut{Aut}
\DeclareMathOperator\h{h}
\DeclareMathOperator\sgn{sgn}
\DeclareMathOperator\arctanh{arctanh}
\DeclareMathOperator\supp{supp}
\DeclareMathOperator\ob{ob}
\DeclareMathOperator\mor{mor}
\DeclareMathOperator\M{M}
\DeclareMathOperator\dom{dom}
\DeclareMathOperator\cod{cod}
\DeclareMathOperator\im{im}
\DeclareMathOperator\Ab{Ab}
\DeclareMathOperator\coker{coker}
\DeclareMathOperator\Cone{Cone}
\DeclareMathOperator\diam{diam}

\makeatletter
\newcommand{\colim@}[2]{%
  \vtop{\m@th\ialign{##\cr
    \hfil$#1\operator@font colim$\hfil\cr
    \noalign{\nointerlineskip\kern1.5\ex@}#2\cr
    \noalign{\nointerlineskip\kern-\ex@}\cr}}%
}
\newcommand{\colim}{%
  \mathop{\mathpalette\colim@{\rightarrowfill@\textstyle}}\nmlimits@
}
\makeatother
%\usepackage{hhline}
%\usepackage{booktabs} 
%\usepackage{array}
%\usepackage{xfrac} 
%\everymath{\displaystyle}
%Enumerate
\usepackage{tikz}
%\usepackgae{graphicx}
\usepackage{subcaption}
\makeatletter
\def\@captionheadfont{}
\makeatother
\renewcommand{\thesubfigure}{\alph{subfigure}}
\usepackage{enumitem} 
%\renewcommand{\labelitemi}{$\bullet$}
%\renewcommand{\labelitemii}{$\ast$}
%\renewcommand{\labelitemiii}{$\cdot$}
%\renewcommand{\labelitemiv}{$\circ$}
%Colors
%\usepackage{color}
%\usepackage[cmtip, all]{xy}
%Main style theorem environment
\newtheoremstyle{main} 		             	 		%Stylename
  	{}	                                     		%Space above
  	{}	                                    		%Space below
  	{\itshape}			                     		%Body font
  	{}        	                             		%Indent
  	{\bfseries\boldmath}   	                         		%Head font
  	{.}            	                        		%Head punctuation
  	{ }           	                         		%Head space 
  	{\thmname{#1}\thmnumber{ #2}\thmnote{ (#3)}}	%Head specification
\theoremstyle{main}
\newtheorem{theorem}{Theorem}[chapter]
\newtheorem{definition}[theorem]{Definition}
\newtheorem{proposition}[theorem]{Proposition}
\newtheorem{corollary}[theorem]{Corollary}
\newtheorem{lemma}[theorem]{Lemma}
\newtheoremstyle{nonit} 		             	 		%Stylename
  	{}	                                     		%Space above
  	{}	                                    		%Space below
  	{}			                     		%Body font
  	{}        	                             		%Indent
  	{\bfseries\boldmath}   	                   		%Head font
  	{.}            	                        		%Head punctuation
  	{ }           	                         		%Head space 
  	{\thmname{#1}\thmnumber{ #2}\thmnote{ (#3)}}	%Head specification
\theoremstyle{nonit}
\newtheorem{remark}[theorem]{Remark}
\newtheorem{examples}[theorem]{Examples}
\newtheorem{example}[theorem]{Example}
\newtheorem{problem}[theorem]{Problem}
\newtheoremstyle{ex} 		             	 		%Stylename
  	{}	                                     		%Space above
  	{}	                                    		%Space below
  	{\small}			                     		%Body font
  	{}        	                             		%Indent
  	{\bfseries\boldmath}   	                         		%Head font
  	{.}            	                        		%Head punctuation
  	{ }           	                         		%Head space 
  	{\thmname{#1}\thmnumber{ #2}\thmnote{ (#3)}}	%Head specification
\theoremstyle{ex}
\newtheorem{exercise}[theorem]{Exercise}
%German non-ASCII-Characters
%Graphics-Tool
%\usepackage{tikz}
%\usepackage{tikzscale}
%\usepackage{bbm}
%\usepackage{bera}
%Listing-Setup
%Bibliographie
\usepackage[backend=bibtex, style=alphabetic]{biblatex}
%\usepackage[babel, german = swiss]{csquotes}
\bibliography{bibliography}
%PDF-Linking
%\usepackage[hyphens]{url}
\usepackage[bookmarksopen=true,bookmarksnumbered=true]{hyperref}
%\PassOptionsToPackage{hyphens}{url}\usepackage{hyperref}
\urlstyle{rm}
\hypersetup{
  colorlinks   = true, %Colours links instead of ugly boxes
  urlcolor     = blue, %Colour for external hyperlinks
  linkcolor    = blue, %Colour of internal links
  citecolor    = blue %Colour of citations
}
\newcommand{\bld}[1]{\boldmath\textit{\textbf{#1}}\unboldmath}
\newcommand{\eqclass}[1]{\sbr[0]{#1}}
\newcommand{\cat}[1]{\mathsf{#1}}
\newcommand{\Sbb}{\mathbb{S}}
\newcommand{\Zbb}{\mathbb{Z}}
\newcommand{\Nbb}{\mathbb{N}}
\newcommand{\Rbb}{\mathbb{R}}
\newcommand{\Hbb}{\mathbb{H}}
\newcommand{\Cbb}{\mathbb{C}}
\newcommand{\Tcal}{\mathcal{T}}
\newcommand{\SLrm}{\mathrm{SL}}
\newcommand{\PSLrm}{\mathrm{PSL}}
\newcommand{\SLrmstar}{\mathrm{S^*L}}
\newcommand{\PSLrmstar}{\mathrm{PS^*L}}
\newcommand{\GLrm}{\mathrm{GL}}
\newcommand{\Mrm}{\mathrm{M}}
\newcommand{\Isom}{\mathrm{Isom}}
\newcommand{\Mob}{\mathrm{M\ddot{o}b}}
\newcommand{\Ebb}{\mathbb{E}}
\newcommand{\Cscr}{\mathscr{C}}
\newcommand{\pwrm}{\mathrm{pw}}
\newcommand{\clos}[1]{\overline{#1}}
\newcommand{\Hcal}{\mathcal{H}}
\newcommand{\Hbcal}{\bm{\mathcal{H}}}
\newcommand{\Ucal}{\mathcal{U}}
\newcommand{\Ubcal}{\bm{\mathcal{U}}}
\renewcommand{\det}{\mathrm{det}}
\newcommand{\ab}{\mathrm{ab}}

\usepackage{trimclip}
\newcommand{\crop}[1]{\clipbox*{0pt -.5ex {\width} {.7\height}}{#1}}
\newcommand{\jota}{\crop{\raisebox{-2pt}{\scalebox{1}[1.5]{\rotatebox[origin=c]{-32}{\reflectbox{$\iota$}}}}}}
